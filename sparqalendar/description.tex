\documentclass[11pt,a4paper]{article}
\usepackage{a4wide,ngerman,url}
\usepackage[utf8]{inputenc}

\parskip4pt
\parindent0pt

\newcommand{\comment}[1]{\par\paragraph{Comment:} #1}

\title{Technical documentation}
\author{Compiled by Hans-Gert Gr\"abe, leipzig-data.de}
\date{Version of Jan 05, 2014}

\begin{document}
\maketitle

\section{Purpose of the Plugin}

This plugin is a first demonstration how directly (not using any intermediate
store) to extract RDF information from a Sparql endpoint and to display it
within Wordpress using the short code mechanism.  State information is stored
by the PHP session mechanism in a JSON object and explored for different
aspects through HTTP GET constructs. The plugin is designed for use with a
Virtuoso based Sparql endpoint.

\section{Background}

Within the Leipzig Data Project \texttt{http://leipzig-data.de} we use
Wordpress as one of the blueprints to display semantically rich information
that is stored in RDF triple stores and can be retrieved via sparql endpoints.
The Wordpress plugin mechanism is best suited to consolidate ideas and share
useful code between different Wordpress instances.

See \texttt{http://www.leipzig-data.de/wordpress} for more information (in
german) about our Wordpress based RDF activities.

This plugin was designed and implemented by Johannes Frey based on earlier
code developed for the Inspirata booking system
\texttt{http://www.inspirata.de/anmeldung}.

\section{General Design}

The plugin consists of the following resources
\begin{itemize}       
\item \texttt{kalender.php} -- at the moment the main engine, should be
  refactored along functional binding rules;
\item \texttt{readme.txt} -- WP plugin description in standard form;
\item \texttt{styles.css} -- CSS styles of the plugin;
\item \texttt{rdfmapping.php} -- a hook for URI reduction into readable
  strings
\item \texttt{sparqalendar.php} -- embedding of the plugin into the overall
  Wordpress structure.

  \comment{Whats about to have here only the overall embedding of the plugin
    into the Wordpress structure and to put the \texttt{add\_shortcode}
    commands in the different display php files? Whats about a common naming
    scheme for display php files? }
\end{itemize}

\section{Design in more Detail}

For the moment all business logic is contained in the script
\texttt{kalender.php} with main functionality:
\begin{itemize}
\item \url{run_query} -- run a sparql query at leipzig-data.de endpoint
  and request the result in json format, type can be 'select' or 'construct.

  Forms an HTTP request from a Sparql query, sends that query via
  \url{wp_remote_get} to the Sparql endpoint to return a json object, decodes
  that object into a PHP array and returns that array.

  \comment{To be refactored into a separate query engine with well designed
    interface.} 

\item \url{fetch_events} -- get event IDs, labels, start and end dates for the
  given month from the Sparql endpoint via a dedicated Sparql query and store
  them per day in a global variable \texttt{\$sp\_events} as arrays (ID,
  label).

  Uses PHP DateTime objects to manage date information. 

\item \url{printEventsPerDay} -- extracts info from \texttt{\$sp\_events} for
  the given day.

\item \url{printEventInfo} -- get full event information for the event ID from
  the Sparql endpoint via a dedicated Sparql query and display it.

\item \url{displayKalenderGet} -- a revised version of the Inspirata calendar
  renderer. 

\item \url{get_param_url} -- retrieve the current requested url and (re)set
  the given parameters in that uri while reusing the parameters of the old
  request, which have not been changed.

  Required for \url{displayKalenderGet} to create additional HTTP Get requests
  from the base URL to display additional information for selected days. 

  \comment{In the early Inspirata code also a POST version existed.  Would it
    be a better idea to store and communicate session state in a POST variable
    at the server side? What are the requirements, benefits and odds of such a
    concept?  Surely odd: The server has to track the user.}

\end{itemize}

\section{Experimental add-ons}

\begin{itemize}
\item \texttt{config.php} -- no idea what this is about;     
\item \texttt{rdfquery.php} -- a first attempt to shape a useful RDF Query
  engine;
\item \texttt{fachgruppe.php} -- display code used at
  \url{http://www.fachgruppe-computeralgebra.de/fachgruppenleitung}.
\end{itemize}

\end{document}
