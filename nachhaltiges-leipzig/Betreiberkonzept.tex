\documentclass[a4paper,11pt]{article}
\usepackage{a4wide,ngerman,url}
\usepackage[utf8]{inputenc}
\parskip4pt
\parindent0pt

\newcommand{\zuklaeren}[1]{\begin{quote} \textbf{Zu klären:} #1 \end{quote}}

\title{Betreiberkonzept der Datenbank\\ „Nachhaltiges Leipzig“ (Entwurf)} 
\author{Hans-Gert Gräbe}
\date{Version vom 25. März 2020}

\begin{document}
\maketitle
\begin{abstract}
  Die Leipziger Nachhaltigkeitsdatenbank (ND) ist ein von RENN.Mitte und der
  Leipziger Zukunftsakademie vorangetriebenes Projekt, um regionale
  Aktivitäten im BNE- und MINT-Bereich in Leipzig zu unterstützen und zu
  popularisieren.
\end{abstract}
\tableofcontents 
\newpage

\section{Zielstellung}

\emph{Nachhaltiges Leipzig}\footnote{\url{https://nachhaltiges-leipzig.de/}.}
ist ein stadtweites Projekt, um Anbietern in den Bereichen Nachhaltigkeit und
MINT eine gemeinsame Plattform zu bieten, über die standardisierte
Informationen zu den \emph{Anbietern} sowie deren \emph{Aktivitäten}
verbreitet werden.  Nach dem Zusammengehen mit \emph{Leipzig
  Grün}\footnote{\url{http://www.leipziggruen.de/}.} sind inzwischen über 300
Anbieter auf der Plattform vertreten. 

Die Anbieter nutzen eine webbasierte Erfassungsschnittstelle, um die
entsprechenden Informationen bereitzustellen und aktuell zu halten.  Die
Plattform stellt eine REST-Schnittstelle zur Verfügung, über welche
Informationen strukturiert ausgelesen und in eigene Web-Dar"|stellungen
eingebunden werden können.

\section{Betreibermodell}

Das Betreibermodell unterscheidet die üblichen drei Servicelevel
\begin{itemize}
\item Level 1: Akteursebene -- verantwortlich für die Aktualität der Inhalte
  der Datenbank,
\item Level 2: Redaktionsebene -- verantwortlich für die Konfiguration der
  Datenbank,
\item Level 3: Plattformebene -- verantwortlich für den Betrieb der
  Plattform. 
\end{itemize}
\paragraph{Level 1}
wird von den Akteuren selbst verantwortet. Akteure haben dafür ein Login als
\emph{User}\footnote{\url{https://daten.nachhaltiges-leipzig.de/}}
eingerichtet und können über die entsprechenden Erfassungsmasken eigene
Aktivitäten verschiedenen Typs eingeben und mit der vorgegebenen
Verschlagwortung kategorisieren.

Logins werden nach längerer Zeit von Inaktivität als inaktiv geschaltet.  Der
Akteur wird darüber vorher informiert und hat Gelegenheit, diesen Prozess
durch einfache Aktion „auf null“ zu setzen.

\zuklaeren{Wie funktioniert das genau.}

\paragraph{Level 2}
wird vom \emph{Betreiberteam} als Admins verantwortet. Sie haben dafür ein
Login als
\emph{Admin}\footnote{\url{https://daten.nachhaltiges-leipzig.de/admin}}.
Aktuell gehören zum Betreiberteam
\begin{center}
  \begin{tabular}{llc}
    Ralf Elsässer & ZAK & seit 19.06.2013\\
    Martin ? & Eine Welt & seit 04.12.2014\\
    Matthias Schirmer & ZAK & seit 15.01.2016\\
    Hans-Gert Gräbe & ZAK & seit 06.09.2017\\
    Matthias Bermninger & Leipziggruen & seit 29.09.2017\\
    Stefan Härtel & appPlant & seit 23.04.2018\\
    Antje Arnold & ZAK & seit 27.02.2019\\
    Markus Schirmer & NN & seit 04.03.2019\\
    ?? praktikum@civixx & NN & seit 17.09.2019\\
    Joschka ?? & afeefa & seit 29.11.2019
  \end{tabular}
\end{center}
\paragraph{Level 3.}
Datenbank und Erfassungssystem wurden ursprünglich von der Firma
\emph{aboutSource}\footnote{\url{https://www.aboutsource.net/}} entwickelt und
betrieben.  Das System ist auf der Basis von Ruby on Rails programmiert. 

2018 wurde das Projekt im Zusammenhang mit einem Weiterentwicklungsauftrag
abgegeben und auf der Basis des übergebenen Quellcodes von der Firma
\emph{appPlant}\footnote{\url{http://appplant.de/}} weiterentwickelt, die das
Ganze seither auch betreibt.

\section{Nutzung der Datenbank}

Informationen der Datenbank werden derzeit an folgenden Stellen genutzt:

\paragraph{Im Veranstaltungskalender des Forums Nachhaltiges Leipzig.}
\begin{center}
  \url{https://www.nachhaltiges-leipzig.de/leben/events}
\end{center}
Dieser Veranstaltungskalender wurde von der Firma \emph{Studio
  Bosco}\footnote{\url{https://www.studiobosco.de}} auf der Basis der
REST-Schnittstelle konzipiert. 

\paragraph{Im Veranstaltungskalender des Zukunftsdiploms.}
\begin{center}
  \url{https://www.zukunftsakademie-leipzig.de/ziele/zukunftsdiplom/veranstaltungen-und-aktionen-zum-zukunftsdiplom/}
\end{center}

Die Webseite in der Wordpress basierten Website der Zukunftsakademie enthält
als experimentelle prototypische Implementierung einen WP-Shortcode, der das
Skript \texttt{content.php} aus der Demo-Version zur ND-Schnittstelle im
Leipzig-Data-Projekt einbindet, siehe das github-Projekt
\texttt{LeipzigData/web}, Verzeichnis \texttt{demo/zd-web}.

Der Code des Plugins ist ebenda im Verzeichnis \texttt{plugins/zukunftsdiplom}
zu finden. 

\section{Datenmodell}

Im Modell werden die zwei Klassen \emph{Akteure} (\texttt{users}) und
\emph{Aktivitäten} (\texttt{activities}) unterschieden, welche über die
REST-API ausgelesen werden können.  Im Modell sind weitere Datenstrukturen
implizit als Konzepte vorhanden.

Das ist zum einen eine genauere Aufteilung eines \emph{Akteurs} in
\begin{itemize}\itemsep0pt
\item die Beschreibung des Akteurs,
\item die Adresse,
\item der Ansprechpartner und
\item die Accountdaten des Ansprechpartners,
\end{itemize}
die alle in einer gemeinsamen Datenbanktabelle \emph{users} zusammengefasst
sind.

\emph{Aktivitäten} sind in verschiedene Aktivitätstypen unterteilt, welche
durch den Wert eines speziellen Attributs \texttt{type} unterschieden werden.
Neben generischen Attributen haben die einzelnen Aktivitätstypen weitere
spezielle Attribute, was man in einer zweistufigen (erweiterbaren)
Vererbungshierarchie oder in einem RDF-Modell bzw. als abstrakten Datentyp (im
Sinne etwa von Java Interfaces) als Mengen von Signaturen beschreiben kann.
Im Weiteren wird die letztere Darstellungsform verwendet. 

Die Klasse \emph{Akteure} umfasst ein Sublimat aus Informationen zu
\texttt{ld:Akteur}, zu dessen Adresse sowie zu Personen, die für den Akteur
als Ansprechpartner tätig sind.  Es lässt sich nicht unterscheiden, ob zum
Beispiel eine Telefonnummer oder eine Adresse zum Vereinsbüro gehört oder zum
Ansprechpartner.  An anderer Stelle wird aber sehr wohl der Datentyp
\emph{Kontakt} („Kontakt zur Abholung“ einer Ressource, „Ansprechpartner“
eines Bildungsangebots) als Teilinterface mit [Name, Email-Adresse, Telefon]
verwendet.

Die Klasse \emph{Aktivitäten} zerfällt in die Unterklassen („Datensatztypen“
in der NL-Sprache) \texttt{Event}, \texttt{Action}, \texttt{Project},
\texttt{Service} und \texttt{Store}, die über das Feld \texttt{type}
unterschieden werden. Im Zuge der Überarbeitung Ende 2018 wurden die „Klassen“
\texttt{Ressource}, \texttt{Bildungsangebot} und \texttt{Beratungsangebot} als
Unterklassen von \texttt{Service} eingeführt, wobei hier die zusätzliche
Unterscheidung über das Feld \texttt{service\_type} ausgeführt wird.

In der aktuellen (Febr 2019) Version werden für einen „einfachen Akteur“
Erfassungsmasken für die Datensatztypen \emph{Project}, \emph{Event},
\emph{Action} sowie die drei Servicetypen \emph{Ressourcen},
\emph{Bildungsangebot} und \emph{Beratungsangebot} angeboten.  Zur
Unterscheidung der Instanzen werden ID's als numerische Primärschlüssel der
Datenbank verwendet.

\paragraph{Adressen.} 
In der Anforderungsbeschreibung für eine Erweiterung (Ende 2017) soll eine
weitere Klasse \emph{Orte} ergänzt werden, die eine akteursübergreifende
Verwaltung von Adressdaten umsetzt. Eine \emph{Adresse} ist dabei eine
implizite Datenstruktur, die über die API als Menge von Attributen
\begin{itemize}\itemsep0pt
  \item full\_address -- String
  \item district -- String
  \item latlng -- Array
\end{itemize}
ausgeliefert wird. \texttt{full\_address} ist dabei bereits ein Aggregat, das
aus den intern separierten Bestandteilen \emph{Straße und Hausnummer},
\emph{PLZ} und \emph{Ort} kombiniert wurde.

Adressen\footnote{In der NL-Terminologie „Orte“, allerdings werden zu Orten in
  der aktuellen Konzeption keine Daten jenseits der Adressdaten und eines
  (verbindlichen) Namens gespeichert.} werden grundsätzlich als Datenaggregate
verwaltet, nicht als Datenobjekte.  Wenn zu einer Aktivität keine Adresse
angegeben ist, wird die Adresse aus den Profildaten des Akteurs übernommen.
Spätere Änderungen dieser Profildaten haben aber aktuell keine Auswirkung auf
diesen Klon.

Es ist eine akteursübergreifende Datenbank zu Orten im Aufbau, aus der häufig
verwendete Adressdaten übernommen werden können.  Auch dies geschieht durch
einfache Übernahme der entsprechenden Attributwerte. Geplant ist, Änderungen
in dieser Datenbasis in die entsprechenden ADressfelder von Aktivitäten (und
auch Akteuren?) zu propagieren, wozu ein System von Backlinks aufgebaut werden
müsste. Unklar bleibt, wie mit zwischenzeitlichen Änderungen umgegangen wird,
die vom Besitzer der Aktivität an den früher übernommenen Adressdaten
vorgenommen worden sind.

\paragraph{Akteure.}
In der Collection \texttt{users} (Akteure) sind Informationen über Akteure
zusammengefasst, wobei nicht zwischen den juristischen Personen und den für
diese agierenden Personen unterschieden wird. 

Prädikate in users.json:
\begin{itemize}\itemsep0pt
\item Akteur
\begin{itemize}\itemsep0pt
  \item id -- String
  \item name -- String, Name der Organisation
  \item organization\_type -- String, Art der Organisation (gewerbliches
    Unternehmen, gemeinnütziger Verein, Stiftung, Genossenschaft, Initiative,
    Freiberufler, Bildungseinrichtung, Sonstige Organisation)
  \item organization\_url -- String, Homepage
  \item organization\_logo\_url -- String, Logo
\end{itemize}
\item ?? -- Checkbox, Handels- oder Gastronomieeinrichtung (nicht mit
  ausgeliefert)
\item ?? -- Checkbox, veröffentlicht (nicht mit ausgeliefert)
\item ?? -- Checkbox, aktiv (nicht mit ausgeliefert)
\item Adresse (des Akteurs oder des Ansprechpartners?)
\item ?? -- Checkbox, Anschrift öffentlich sichtbar (nicht mit ausgeliefert)
\item Ansprechpartner
\begin{itemize}\itemsep0pt
  \item first\_name -- String
  \item last\_name -- String
  \item organization\_position -- String
\end{itemize}
\item Account- und Erreichbarkeitsdaten
\begin{itemize}\itemsep0pt
  \item email -- String
  \item phone\_primary -- String
  \item phone\_secondary -- String
  \item Passwort (nicht mit ausgeliefert)
\end{itemize}
\end{itemize}

\paragraph{Aktivitäten.}
\texttt{activities} ist ein Obertyp zu verschiedenen Arten von Aktivitäten
(Aktionen, Events, Projekte, Services, Stores, \ldots), die mit dem Prädikat
\texttt{nl:hasType} näher spezifiziert werden.  In der Collection
\texttt{activities} sind Informationen über die verschiedenen Typen von
Aktivitäten zusammengefasst, wobei nicht alle Prädikate bei allen Untertypen
verwendet werden. 

Generische Prädikate:
\begin{itemize}\itemsep0pt
  \item id -- String
  \item type -- String (Typ der Aktivität), Auswahl
  \item user\_id -- String (Id des beteiligten Akteurs), Auswahl
  \item name -- (Titel, Name) String
  \item description -- (Beschreibung) String
  \item Adresse (an der die Aktivität stattfindet) -- Ortsauswahl
  \item is\_fallback\_address -- String (Boolescher Wert, Bedeutung unklar)
  \item info\_url -- String
  \item video\_url -- String
  \item image\_url -- String
  \item categories -- Array, weitere Kategorien (siehe Kategorienkonzept)
  \item first\_root\_category -- String, Hauptkategorie
\end{itemize}

Weitere Prädikate für Actions:
\begin{itemize}\itemsep0pt
  \item start\_at -- String, Zeitraum/Termine (als Freitextfeld)
\end{itemize}

Weitere Prädikate für Events:
\begin{itemize}\itemsep0pt
  \item start\_at -- String (mit Auswahl Datum/Zeit hinterlegt)
  \item end\_at -- String (mit Auswahl Datum/Zeit hinterlegt)
  \item target\_group -- (Zielgruppe) String, Freitextfeld
  \item costs -- (Kosten) String, Freitextfeld
  \item requirements -- (Bedingungen) String, Freitext-Area
  \item speaker -- (Referenten) String, Freitextfeld
  \item ?? -- Checkboxen, kostenfrei, kinderfreundlich, barrierefrei (nicht
    mit ausgeliefert) 
  \item goals -- (Ziele) Array, Mehrfachauswahl, wird aktuell nicht mit
    ausgeliefert.
\end{itemize}

Weitere Prädikate für Projects:
\begin{itemize}\itemsep0pt
  \item short\_description -- (Kurzbeschreibung) String, Freitext-Area
  \item goals -- (Ziele) Array, Mehrfachauswahl
  \item property\_list -- Array, Liste mit speziellen Merkmalen, als
    Freitext-Area, die zeilenweise ausgelesen wird. 
\end{itemize}

Weitere Prädikate für Services:
\begin{itemize}\itemsep0pt
  \item target\_group -- String, Zielgruppenbeschreibung
  \item costs -- String, Kosten
  \item requirements -- String, Bedingungen
  \item short\_description -- String, Kurzbeschreibung
  \item goals -- Array
  \item service\_type -- String, Angebotsart (Workshop, Exkursion, Vortrag,
    GTA, Unterrichtseinheit, Beratungsangebot)
  \item target\_group\_selection -- String, Auswahl (Kindergarten,
    Grundschule, Sekundarstufe, Erwachsenenbildung)
  \item duration -- String
\end{itemize}
Neu gibt es Beratungsangebote und Bidungsangebote, jeweils ohne Feld
„Angebotsart“.

Weitere Prädikate für Store:
\begin{itemize}\itemsep0pt
  \item short\_description -- String
  \item property\_list -- Array
  \item products -- Array
  \item trade\_categories -- Array
  \item trade\_types -- Array
\end{itemize}

Auf dieser Basis sind folgende Transformationen nach LDD möglich:
\begin{itemize}\raggedright
\item \texttt{full\_address} kann als \texttt{ld:proposedAddress} in eine
  syntaktisch korrekte URI einer \texttt{ld:LeipzigerAdresse} transformiert
  werden. 
\item \texttt{latlng} kann in eine \texttt{gsp:asWKT} Geo-Adresse
  transformiert werden.
\end{itemize}

\paragraph{Weitere Teile der Modellierung.}
Diese sind noch wenig ausgearbeitet und enthalten oft nur wenige Instanzen pro
Klasse. 
\begin{itemize}\itemsep0pt
\item \texttt{categories} repräsentiert eine baumartige Struktur verschiedener
  Tags, die einzelnen Aktivitäten zugewiesen sind.  Diese Struktur wird zur
  Menüführung verwendet. 
\item \texttt{goals} repräsentiert eine geordnete Liste verschiedener Tags, die
  einzelnen Aktivitäten zugewiesen sind. Das ist als Tagwolke modelliert, kann
  aber -- mit Blick auf die Datenqualität -- nur von einem Administrator
  erweitert werden.
\item \texttt{products} repräsentiert eine Liste verschiedener
  Produktkategorien, die einzelnen Stores zugewiesen sind.
\item \texttt{trade\_types} und \texttt{trade\_categories} repräsentieren zwei
  geordnete Listen verschiedener Tags, die einzelnen Akteuren über 
  Crossreferenz-Tabellen zugewiesen sind.
\end{itemize}

\end{document}
