\documentclass[11pt,a4paper]{article}
\usepackage{a4wide,ngerman,url}
\usepackage[utf8]{inputenc}

\makeatletter
\let\@listI\@listii

\parskip4pt
\parindent0pt

\begin{document}
\begin{center}
  \textbf{\Large Technische Dokumentation\\[.3em] des Leipzig Data Events
    Widgets}\\[1em] Version vom 12. September 2013
\end{center}

\section{Allgemeines}
Ziel der entwickelten Softwarelösung ist die prototypische Web-Darstellung von
Event- und begleitenden Informationen aus dem Leipzig Open Data Projekt [LOD]
auf Webseiten angeschlossener \emph{Akteure} (siehe Glossar) auf der Basis des
JS-Frameworks \emph{Exhibit}.

Die Lösung bietet dem \emph{Webdienste-Anbieter} die Möglichkeit, auf einfache
Weise eine lesende Sicht auf vorhandene Daten mit Such- und Filterfunktionen
zu definieren und in die Webpräsenzen der von ihm betreuten Akteure
einzubauen.  Der Webdienste-Anbieter kann dazu relevante Daten aus
Datenbeständen, die nach einer vorgegebenen Ontologie aufgebaut sind, der
\emph{LD.Events Datenbasis}, in einen eigenen lokalen ARC2 Data Store
auslesen, daraus über Filterkriterien akteursspezifisch einen relevanten
Teilbestand für die Präsentation auswählen und in die Webpräsenz des Akteurs
als iframe einbinden.  Nutzer der Webpräsenz des Akteurs können in diesen
Informationen weiter selektiv suchen und sich die Ergebnisse in verschiedenen
Formaten anzeigen lassen.

Alternativ kann die LD.Events Datenbasis oder Teile davon auch direkt über den
Leipzig Data Sparql-Endpunkt \url{http://www.leipzig-data.de:8890/sparql}
ausgelesen werden. 

Nicht erfasst ist der Prozess des Fortschreibens der LD.Events Datenbasis
durch einen Kreis dafür autorisierter Personen -- hierfür ist der bisherige
Datenerfassungsprozess auf Protokollebene so zu harmonisieren, dass relevante
und qualitätsgesicherte Daten direkt in die LD.Events Datenbasis übernommen
werden können. In einer weiteren Ausbaustufe kann hier ein Mashup aus
verschiedenen Kanälen eingebaut werden, die dem vereinbarten Protokoll folgen.
Derzeit können Veranstaltungsdaten von autorisierten Personen direkt über das
Ontowiki-Portal \url{http://leipzig-data.de/Data} des Leipzig Data Projekts
eingegeben werden.

Die Softwarelösung basiert auf Javascript, die Präsentation läuft komplett mit
allen Such- und Darstellungsfunktionen im Webbrowser des Nutzers ab und greift
auf ein JSON-Datenobjekt zu, das mit der Javascript-Funktionalität des
Widget-Framework \emph{Exhibit} weiter aufbereitet wird.

Die Softwarelösung wurde von \textbf{Johannes Frey} im Rahmen des Leipzig Open
Data Projekts [LOD] entwickelt. Das Projekt wurde durch die Stadt Leipzig im
Rahmen der Open Innovation Ausschreibung 2012 unterstützt.

\section{Produktübersicht}

\subsection{Übersicht}

Über eine SPARQL-Anfrage kann der Webdienste-Anbieter regelmäßig relevante
Daten aus der LD.Events Datenbasis auslesen und mit dem PHP Skript
\texttt{Store.php} als Dump in einen eigenen lokalen ARC2 Data Store
speichern oder direkt weiter verarbeiten. 

Aus diesem Datenbestand können Webdienste-Anbieter akteursspezifische Angebote
zusammenstellen.  Dazu müssen diese pro Akteur ein eigenes \emph{Thema}
erstellen oder aus vorgegebenen Themen auswählen oder anpassen, in welchem die
akteursspezifischen Vorgaben und Wünsche umgesetzt sind.  Die
Anpassungsmöglichkeiten beziehen sich auf
\begin{itemize}
\item die Auswahl des relevanten Grunddatenbestands (über einen gestaltbaren
  PHP-Filter\-prozess \texttt{getdata.php}, über den auch regelmäßig die Daten
  aktualisiert werden),
\item Umfang und Anordnung der Such- und Filterfunktionen, die den Nutzern zur
  Verfügung stehen (über eine Exhibit-Template-Datei
  \texttt{presentation.php}, die akteursspezifisch angepasst werden kann), und
\item das Layout (über eine akteursspezifische CSS-Datei \texttt{styles.css}). 
\end{itemize}
Im Zentrum des Konzepts steht das Javascript basierte Widget-Framework
\emph{Exhibit} in der Version 2.0. In den weiteren Ausführungen wird die
Kenntnis dieses Frameworks im Umfang von [Ex] vorausgesetzt.

\subsection{Installation}

\begin{itemize}
\item Ausrollen der Dateien in ein eigenes Verzeichnis \texttt{widget} auf dem
  Webserver, das der Webserver lesen kann,
\item Ggf.\ Ausrollen von ARC2, siehe [Arc2], das als Unterverzeichnis
  \texttt{arc2} an \texttt{widget} zu linken ist, Einbinden der Datei(en) im
  Verzeichnis \texttt{plugins} in das Verzeichnis \texttt{arc2/plugins} (etwa
  durch Linken), Aufsetzen einer Datenbank und Konfiguration des ARC2 Data
  Store.

  Dazu ist die Datei \texttt{db\_credentials.php} in eine Datei
  \texttt{db.php} zu kopieren und die Zugangsdaten zur lokalen Datenbank
  einzutragen. In dieser Datenbank werden eine Reihe neuer Tabellen mit dem
  Namenspräfix\footnote{Dieser Präfix kann in der Datei \texttt{Store.php}
    geändert werden.} \texttt{data\_store} angelegt.
\item Auswahl oder Erstellen eines Themas pro Akteur zur Datenpräsentation und
  Einbinden -- z.\,B.\ als Webseite -- in die Struktur der Website.  Dazu kann
  aus den vorhandenen Themen (Verzeichnis \texttt{themes}) ausgewählt werden. 
\end{itemize}
Die Anzeige kann durch Ändern der Präsentations- und Stildateien des Themas
akteurs\-spezifischen Bedürfnissen angepasst werden, was aber einige
Vertrautheit mit dem Exhibit-Framework voraussetzt.  Siehe dazu ebenfalls
[Ex].

\section{Grundsätzliche Struktur- und Entwurfsprinzipien}

\subsection{Einspielen der Event-Daten}
Um die Daten in den durch die vorherigen Schritte eingerichteten ARC2-Store zu
importieren gibt es \textbf{2 Möglichkeiten}.
\begin{enumerate}
\item \textbf{Import mittels SPARQL-Anfrage}

  Dies ist die vorkonfigurierte, einfacherere Variante. In der Datei
  \texttt{Store.php} befindet sich die Funktion \texttt{loadDataFromEndpoint}.
  In dieser kann die SPARQL-Anfrage angepasst werden, um eine Teilmenge der
  Event-Daten aus unserem öffentlichen
  Sparql-Endpunkt\footnote{\url{http://leipzig-data.de/widget/sparql.php}}
  abzufragen. In der Vorkonfiguration werden alle Daten übernommen. Durch
  einfaches Ausführen der Datei \texttt{Store.php} findet der Import
  automatisch statt.

\item \textbf{Import mittels Dateidump}

Dies ist die komplexere, aber stabilere Lösung.  An die unter
\url{http://leipzig-data.de/Data} öffentlich verfügbare LD.Events Datenbasis
kann eine eigene SPARQL-Abfrage\footnote{Über deren Sparql-Endpunkt
  \url{http://leipzig-data.de:8890/sparql}.} gestellt werden, um Event-Daten
zu extrahieren.  Beispiele für solche Anfragen finden sich in der Datei
\texttt{Queries.txt} der Distribution.  Das Ergebnis dieser Abfrage ist in
einer Datei \texttt{EventsDump.ttl} zu speichern und in das Widget-Verzeichnis
zu kopieren. Abschließend muss der Aufruf der Funktion
\texttt{loadDataFromEndpoint} durch \texttt{loadDataFromFile} ersetzt und
\texttt{Store.php} ausgeführt werden.
\end{enumerate}
Eigene Daten können zusätzlich importiert werden, indem eine der eben
genannten Funktionen erweitert wird, unter der Voraussetzung, dass die Daten
in einem RDF-Format vorliegen.

\subsection{Anlegen der json-Datei}

Die Datei \texttt{data.json} enthält alle relevanten Informationen im
Exhibit-spezifischen JSON-Format und ist für das gewählte Thema mit dem
dortigen PHP-Skript \texttt{getdata.php} zu erzeugen.  Hierbei kann eine
weitere nutzerspezifische Datenauswahl durch den Webdienste-Anbieter erfolgen,
um z.\,B.\ in verschiedenen Themen unterschiedliche Daten darzustellen. Dazu
muss die SPARQL-Anfrage in der Funktion \texttt{filterData} verändert werden.

Für das Konvertieren von RDF nach JSON wurden die
RDF-Management-Funktionalitäten von ARC2 um ein
Exhibit-JSON-Plugin\footnote{Siehe die Datei
  \url{plugins/ARC2_ExhibitJSONSerializerPlugin.php} in der Distribution.}
erweitert, welches die Serialisierung des geparsten RDF-Graphen in das
Zielformat realisiert.  Dabei wird eine automatische Erkennung der
Exhibit-Datentypen vorgenommen, sodass z.\,B.\ ein Datum auch den Datentyp
\texttt{date} zugewiesen bekommt.

Die Struktur der Exhibit-JSON-Datei und die verwendeten Datentypen sind in
[Ex] genauer beschrieben.
\begin{quote}
  \emph{Hinweis:}\ Für das korrekte Erstellen von \texttt{data.json} muss das
  PHP-Skript über {Schreibrechte} in dem Verzeichnis verfügen.
\end{quote}

\subsection{Anzeige der Daten}

Zur Anzeige der Daten wird das JSON-Datenobjekt aus der json-Datei geladen und
mit den Layout-Informationen der Webseite über verschiedene Tag-Properties
zusammenführt.

Neben der json-Datei und \texttt{exhibit-api.js} werden die Erweiterungen
\texttt{time-extension.js} und \texttt{calendar-extension.js} verwendet.
Das json-Datenobjekt wird über eine Link-Header Deklaration mit
\texttt{rel={\dq}exhibit/data{\dq}} eingebunden. 

Zur Aktivierung der Google Maps Kartenfunktion wird ein eigener Google Maps
Schlüssel benötigt. Dieser ist in die Datei \texttt{gmaps\_api\_key.php}
einzufügen.

Die Anzeige auf der Webseite wird von Exhibit entsprechend der angegebenen
\texttt{ex:\ldots} Attribute verschiedener \texttt{<div>} Tags gesteuert.  So
lassen sich auf einfache Weise verschiedene Strukturelemente kombinieren, über
die unterschiedliche Sichten auf die Datenauswahl eingestellt und
zusammengeführt werden können.  Die ausgewählten Daten werden dann im
Hauptfeld angezeigt.

\subsection{Aktualisierung}
Es sind die Schritte 3.1 sowie 3.2 erneut auszuführen, um neue Event-Daten
einzuspielen.
\begin{quote}
  \emph{Hinweis:}\ Sollten anschließend in der Präsentation keine
  Veränderungen zu sehen sein, so hilft meist das {Löschen des Browser-Caches}
  weiter.
\end{quote}
\section{Testen}

Zu ergänzen.

\section{Glossar}

\subsection{Begriffe}

\paragraph{Akteur.} 
Anbieter realweltlicher Events, die im verteilten Kalendersystem erfasst
werden sollen, um die Zielgruppe dieses Anbieters auf das Event aufmerksam zu
machen.  Typischerweise ein Verein oder anderweitiger Betreiber eines Orts, an
dem regelmäßig oder unregelmäßig Events stattfinden, der einen
\emph{Webdienste-Anbieter} mit Design und technischer Unterstüt\-zung beim
Betrieb der eigenen Webpräsenz beauftragt hat (oder diese Kompetenz selbst
vorhält).

\paragraph{ARC2.} 
Flexibles RDF-System für PHP-basierte Semantic Web Anwendungen, siehe [Arc2].
Für diese Anwendung muss das mitgelieferte JSON-Plugin installiert werden, das
die erforderlichen Daten im Exhibit-spezifischen Ausgabeformat formatiert. 

\paragraph{Exhibit.} 
Javascript basiertes Publikations-Framework für datenreiche interaktive
Webseiten, siehe \url{http://www.simile-widgets.org/exhibit}.  In dieser
Anwendung wird Exhibit 2.0 verwendet.

\paragraph{Nutzer.} 
Besucher der Website des Akteurs. Restriktionen (etwa in der
Behindertengerechtheit der Anzeige) ergeben sich aus dem verwendeten
Framework, das vom Browser des Nutzers unterstützt werden muss (und im
Normalfall auch unterstützt wird). Der Nutzer muss Javascript aktiviert haben.

\paragraph{OntoWiki.} 
Projekt der AKSW-Gruppe am Institut für Informatik der Universität Leipzig,
siehe \url{http://ontowiki.net/Projects/OntoWiki}.

\paragraph{Thema.} 
Akteursspezifische Anpassung des Frameworks, bestehend aus 
\begin{itemize}
\item einem PHP-Filterskript \texttt{getdata.php},
\item einem Exhibit-Präsentationstemplate \texttt{presentation.php},
\item einer Stylesheet-Datei \texttt{styles.css},
\item einer mit \texttt{getdata.php} zu generierenden und regelmäßig zu
  aktualisierenden JSON-Datendatei \texttt{data.json} mit einem
  akteursspezifische Ausschnitt aus den Grunddaten,
\item ggf einem eigenen Google Maps Schlüssel \texttt{gmaps\_api\_key.php},
\end{itemize}
die vom Webdienste-Anbieter akteursspezifisch aus mehreren Vorlagen
auszuwählen oder selbst herzustellen oder anzupassen ist.

\paragraph{Webdienste-Anbieter.} 
Betreiber einer oder mehrerer Websites von \emph{Akteuren} mit Zugriff auf das
Dateisystem des Webservers, um dort weitere PHP-Funktionalität zu integrieren.
Neben dem Betrieb der Websites leistet der Webdienste-Anbieter \emph{first
  level support} im Bereich der Schulung der Datenverantwortlichen auf
Akteursseite und bietet Beratung beim Design der Website des Akteurs an.

\paragraph{LD.Events Datenbasis.} 
Datenbasis des Leipzig Open Data Projekts [LOD], auf die über eine
SPARQL-Schnittstelle oder ein git-Repo zugegriffen werden kann.
Perspektivisch soll diese Datenbasis zu einer verteilten gemeinsamen
Datenbasis der am Projekt beteiligten Web\-dienste-Anbieter weiterentwickelt
werden, auf die nach Open Data Prinzipien zugegriffen werden kann.

\section{Quellen}

\begin{description}
\item[{[Arc2]}] Flexible RDF system for semantic web and PHP practitioners.\\ 
\url{https://github.com/semsol/arc2/wiki}

\item[{[Ex]}] Getting Started with Exhibit. \\
\url{http://simile-widgets.org/wiki/Getting_Started_with_Exhibit}

\item[{[LOD]}] Das Leipzig Open Data Projekt. Nov. 2012 -- April 2013. \\
\url{http://leipzig-netz.de/index.php5/LD.OpenInnovation-12}

\end{description}

\end{document}
