\documentclass[a4paper,11pt]{article}
\usepackage{a4wide,ngerman,url}
\usepackage[utf8]{inputenc}
\parskip4pt
\parindent0pt

\title{Datenmodelle verschiedener Leipziger Plattformen.\\ Ein Vergleich} 
\author{Hans-Gert Gräbe}
\date{Version vom 10. Oktober 2018}

\begin{document}
\maketitle
\tableofcontents 
\newpage

\section{Vorbemerkungen}

Innerhalb einer Stadt wie Leipzig gibt es viele Akteure mit eigenen Portalen.
Eine Zusammenführung von Informationen erfolgt teilweise auf der Ebene von
Multiplikatoren wie den „Leipziger Ecken“ im Leipziger Osten, die das
Zusammenspiel von Akteuren innerhalb einzelner Stadtteile koordinieren und
eine gemeinsame Stadtteilplattform\footnote{\url{http://leipziger-ecken.de}.}
betreiben.  Weiterhin gibt es stadtweite Plattformen mit speziellen
„Sammelgebieten“\footnote{Etwa \url{https://nachhaltiges-leipzig.de} oder
\url{https://leipzig.afeefa.de}.}.

Eines der Ziele des \emph{Leipzig Data
  Projekts}\footnote{\url{http://leipzig-data.de/}.} besteht darin, einen
\textbf{Leipziger Open Data Raum} voranzubringen, indem verschiedene dieser
Leipziger Portalanbieter miteinander vernetzt werden. Ein erfolgversprechendes
Vorgehen wurde mit den Partnern \emph{Leipziger Ecken} sowie
\emph{Nachhaltiges Leipzig} entwickelt:
\begin{itemize}
\item [1)] Die Partner betreiben eigene Portale zur Datenerfassung und
  organisieren den dazu erforderlichen techno-sozialen Prozess einschließlich
  einer gewissen Qualitätssicherung (insbesondere Spamvermeidung) in ihrer
  jeweiligen Zielgruppe. 
\item [2)] Über eine Schnittstelle (REST, JSON, RDF, \ldots) wird lesender
  Zugriff auf einen definierten teil der Daten gewährt.
\item [3)] Diese Schnittstellen werden im Kontext von \emph{Leipzig Data}
  integriert, entsprechende Metadaten und Dienste entwickelt, um einen
  einheitlichen \emph{verteilten} Datenraum zu konstituieren, der über diese
  integrierte Schnittstelle angemessen semantisch exploriert werden kann. 
\end{itemize}
Die Schnittstellen der einzelnen Partner greifen üblicherweise auf eine lokale
Datenbank zu, welche die lokal verwalteten Zustände (also genaue Informationen
über einzelne Akteure, Orte, Events, \ldots\ auf Instanzebene) einmal für die
eigene Webdarstellung verwaltet und zum anderen für den Austausch vorhält. 

Jeder solchen Datenbank liegt ein eigenes \emph{Datenbankschema} und damit
letztlich ein eigenes \emph{Datenmodell} zugrunde, das im Rahmen der
Integrationsbemühungen zunächst zu explizieren und mit den Datenmodellen der
anderen Partner zu vergleichen ist. 

Mit dieser Zusammenstellung wird ein erster Schritt in dieser Richtung zur
Integration verschiedener Leipziger Plattformen gegangen.  Die Darstellung und
Vergleiche orientieren sich dabei am Datenmodell von Leipzig Data, da dieses
von Anfang an unter dem Blcikwinkel der Interoperabilität entwickelt wurde.
Für die einzelnen Anwendungen werden deshalb vor allem die Differenzen zum
Leipzig Data Datenmodell hervorgehoben. 
\newpage

\section{Leipzig Data}

\subsection{Hintergrund}

Die \emph{Leipziger Initiative für Offene Daten} ist angetreten, um die
Bemühungen zur Etablierung Offener Daten als wesentlichen Teil einer sich
entfaltenden Weblandschaft in der Leipziger Region voranzubringen.

Kern der Bemühungen ist die Etablierung von \emph{Leipzig Data} als einer
signifikaten Menge von Beschreibungen des „Leipziger Lebens”, die unter einer
freien Lizenz in digital adressierbarer Form öffentlich verfügbar sind.

Im Fokus steht allerdings nicht so sehr \emph{Open Data} als vielmehr
\emph{Free Speech}, da wir offene Daten nicht als Selbstzweck begreifen,
sondern als Voraussetzung der Selbstermächtigung mündiger Bürger, Freie Rede
über die sie betreffenden Angelegenheiten zu führen.

Die Initiative setzt die Aktivitäten von API Leipzig (2008-2012) mit
veränderter Schwerpunktsetzung fort und wurde von der Stadt Leipzig in einem
Kurzzeitprojekt (Nov. 2012 bis April 2013) im Rahmen ihrer „Open Innovation”
Ausschreibung unterstützt.
 
\emph{Leipzig Data} betreibt folgende Infrastruktur:
\begin{itemize}\itemsep0pt
\item Webseiten \url{http://leipzig-data.de} auf Wordpress-Basis,
\item Ein github Organisationsaccount \url{https://github.com/LeipzigData} mit
  den Projekten 
  \begin{itemize}\itemsep0pt
  \item \texttt{RDFData} – RDF-Wissensbasen (als Primärdaten)
  \item \texttt{Tools} – eine Reihe von Werkzeugen, im Wesentlichen zum
    Anschauen, wie es gehen könnte (Blick über die Schulter auf die Werkbank), 
  \item \texttt{web} – die Webseiten des Projekts. Der Code kann studiert
    werden, wenn es um die Einbindung von RDF-Quellen in Websites geht.
  \item sowie einer Zahl von weiteren Repos für Teilprojekte.
  \end{itemize}
\item Dazu wird eine \emph{Leipzig
  Ontology}\footnote{\url{http://leipzig-data.de/ontology/}.} entwickelt.
\item Unter \url{http://leipzig-data.de:8890/sparql} wird ein Sparql Endpunkt
  für Queries auf den Daten angeboten, die in einem Virtuoso basierten
  RDF-Store (als Sekundärdaten) gehostet sind.
\item Unter \url{http://leipzig-data.de/info} wird eine Infoseite mit
  Beispielen betrieben, in denen demonstriert wird, wie sich Webseiten aus
  RDF-Quellen erzeugen lassen.  Die Quellen dieser Anwendungen sind im Repo
  \url{https://github.com/LeipzigData/web} zu finden. 
\item Dazu wird eine \emph{Leipzig
  Ontology}\footnote{\url{http://leipzig-data.de/ontology/}.} entwickelt.
\end{itemize}

\subsection{Datenmodell}

Das Datenmodell ist hier nur bis zu einer groben Granularitätsebene
beschrieben.  Für weitere Informationen wird auf
\url{http://leipzig-data.de/ontology} verwiesen.

\subsubsection*{Allgemeine Übersicht über den Datenbestand}

Kern dieses Datenbestands sind aktuell \emph{Akteure}, \emph{Orte},
\emph{Adressen}, \emph{Treffpunkte} und \emph{Events}, die zusammen ein System
von \emph{White Pages} bilden, mit denen Geolokalität auf einheitliche Weise
referenzierbar (und damit auch aufeinander beziehbar) wird. Basis dieses
Systems sind die aus dem API-Leipzig Projekt und damit letztlich von der Stadt
Leipzig übernommenen und weiter aktualisierten über 65\,000 Datensätze
Adressdaten, von denen über 63\,000 mit Geokoordinaten (Übernahme aus
Nominatim mit anschließender weiterer Konsolidierung und Qualitätssicherung)
versehen sind. Damit lassen sich auch ohne Dienste wie Google Maps Leipziger
Orte an Geodaten binden und so auf Karten lokalisieren. Die höhere Qualität
gegenüber eine reinen Referenzierung der Geokoordinaten etwa über die
Google-API ergibt sich aus der vorgenommenen Disambiguierung, die etwa dem
Unterschied zwischen der Verwendung von Rechneradressen und Rechnernamen
entspricht, sowie der Möglichkeit, an diese Adress- oder Orts-URIs weitere
Information zu binden. Diese White Pages werden vor allem im Leipzig Data
Event Projekt fortgeschrieben.

Daneben gibt es Einzelprojekte, mit denen Daten aus verschiedenen Quellen
aufbereitet worden sind wie MINT-Orte, Schulen, Polizeidirektion,
Seniorenbüros.

\subsubsection*{RDF-Graphen und Klassen}

\paragraph{Adressen und Geodaten der Stadt Leipzig.} 
Eine \emph{Leipziger Adresse} ist ein geolokaler Punkt in der Stadt Leipzig
mit einer Hausnummer, wo zum Beispiel eine Postzustellung möglich ist. Die
Daten (über 65\,000 Datensätze) wurden 2012 im Rahmen des API-Leipzig Projekts
von der Stadtverwaltung (einschließlich Referenzen auf das Straßenverzeichnis)
übernommen, im Rahmen des Leipzig Data Projekts in das RDF-Format
transformiert und in \texttt{Adressen.ttl} zusammengefasst.

Hausnummern sind \emph{Grundstücken} zugewiesen, diese können aus mehreren
\emph{Flurstücken} bestehen. Verschiedene Gebäude auf einem solchen Grundstück
könnten später durch URIs bezeichnet werden, die die Grundstücksadresse als
Namenspräfix haben. Damit folgt unsere Bezeichnung der im Kataster.

Im Rahmen des Linked Geodata Projekts wurden diese Daten mit Geodaten
angereichert. Die Daten wurden initial von Claus Stadler über nominatim aus
Open Streetmap extrahiert, danach weiter ergänzt und aktualisiert.

Ausgewählte Adressen außerhalb von Leipzig sind in
\texttt{WeitereAdressen.ttl} nach demselben URI-Schema in einer verkürzten
Ontologie erfasst.

Es gibt geolokale Punkte, die nicht durch eine solche Adresse referenziert
werden können, wie Treffpunkte, Kinderspielplätze u.ä. Dafür wurde das Konzept
\emph{Treffpunkt} eingeführt, das aus einem Bezeichner und weiterer geolokaler
Information besteht.

In Anwendungen (etwa den Events) sind die standardisierten Adressen bis auf
die Hausnummer aufgelöst. Weitere Informationen sind als Adresszusatz
(\texttt{ld:hasAddressAddendum}) einzutragen.

Die URI einer solchen Adresse wird nach einheitlichen Prinzipien aus den
Informationen (PLZ, Stadt, Straße, Hausnummer) aufgebaut, so dass auch aus
anderen Adresssystemen diese URIs generieret werden können, insofern diese
vier Datenbestandteile separiert werden können.  

\textbf{Prädikate:}
\begin{itemize}\itemsep0pt
\item ld:hasPostCode Literal – Postleitzahl
 \item ld:hasCity Literal – Stadt
\item ld:hasStreet Literal – Straße, in welcher sich die Adresse befindet
\item rdfs:label Literal – Bezeichner, etwa “Leipzig, Messering 6” 
\item gsp:asWKT Literal – Geokoordinaten im WKT-Format “Point(long lat)”
\item ld:inOrtsteil ld:Ortsteil – Ortsteil, in welchem sich die Adresse
  befindet
\item ld:inStreetId ld:Strasse – Straße, in welcher sich die Adresse befindet
\end{itemize}

\paragraph{Orte.} 
Orte in der Stadt Leipzig sind Orte mit einer \emph{Adresse} und einem
\emph{Träger}, an denen Events veranstaltet werden.

Neben dieser Art von Orten existieren auch Stellen in Leipzig, die einen
gebräuchlichen Namen haben, aber nur durch Geokoordinaten referenzierbar sind,
also zur Kategorie Treffpunkt gehören. Dazu gehören Spielplätze, die aus
Stadtdaten übernommen wurden, die codefor.de/leipzig (Stand 2014) extrahiert
hat.

\textbf{Prädikate:}
\begin{itemize}\itemsep0pt
\item dct:modified xsd:date – letzte Modifikation des Datensatzes
\item Orga-Literale foaf:mbox, foaf:phone, foaf:homepage
\item ld:erreichbar Literal – Erreichbarkeit mit öffentlichem Nahverkehr
\item ld:hasAddress ld:LeipzigerAdresse – Adresse
\item ld:hasAddressAddendum Literal – Adresszusatz
\item ld:hasAnschrift Literal – Anschrift, Postfach oder so, wenn der Ort
  keine ld:Adresse hat (Beispiel: LSGM)
\item ld:hasSupplier org:Organization oder Unterklasse – Träger
\item ld:hasTag ld:Tag – Klassifizierung (Konsolidierung der literalen Werte
  von ld:Art und ld:Bereich)
\item ld:contactPerson, ld:engagedPerson ld:Person – am Ort engagierte
  Personen
\item rdfs:label Literal – Bezeichnung
\item Inhalts-Literale (alle mehrfach möglich) ld:Arbeitsformen, ld:Art,
  ld:Auszeichnungen ld:Finanzierung, ld:Hintergrund, ld:Kosten,
  ld:Kurzinformation, ld:Leistungsangebot, ld:Oeffnungszeiten,
  ld:Teilnahmebedingungen, ld:Zielgruppe, ld:Zielstellung
\item Einordnungsliterale zu einer der städtischen Übersichten: ld:Categories,
  ld:hasStadtId
\end{itemize}

\paragraph{Akteure und Personen.} 
Synonym wird auch der Begriff \emph{Träger} verwendet. \emph{Akteure} sind
natürliche oder juristische Personen mit verschiedenen Aktivitäten. Diese
wurden zunächst in einem RDF-Graphen \texttt{Traeger.ttl} zusammengefasst,
sind aber nun nach verschiedenen Kriterien in einzelne RDF-Graphen
aufgesplittet.  Die Ontologie orientiert sich an der org-Ontologie, die für
die verschiedenen Unterklassen um einzelne Felder erweitert wird.

\emph{Juristische Personen} sind rechtsfähige Träger verschiedener Aktivitäten
und Betreiber von Orten in Leipzig. Eine juristische Person ist eine
Unterklasse von \texttt{org:Organization} und kann weiteren Klassen wie
\texttt{ld:Verein}, \texttt{ld:Unternehmen} zugeordnet werden, soll aber immer
auch \texttt{org:Organization} sein, um Inferenz längs Vererbungshierarchien
zu vermeiden. URIs juristischer Personen haben die Gestalt
\texttt{Data/<OrgForm>/<name>}, wobei \texttt{<OrgForm>} auf die
Organisationsform hinweist. Damit soll diese Information perspektivisch
verfeinert werden.

Personen sind im RDF-Graphen \texttt{Personen.ttl} als \texttt{foaf:Person}
erfasst und in den anderen RDF-Graphen referenziert, etwa über das
Prädikat\texttt{org:hasMember}.

Über das Prädikat \texttt{owl:sameAs} werden Verweise auf dieselben Akteure
oder Personen in den Datenbanken von Partnern verwaltet. 

\paragraph{Events.}
Ziel dieses Teilprojekts ist es, eine Infrastruktur aufzubauen, in die
Event-Daten in einheitlichem Format aus verschiedenen Quellen und von
verschiedenen Akteuren eingespeist werden und der Allgemeinheit zum Gebrauch
zur Verfügung stehen.

Die Infrastruktur bietet keinen elaborierten eigenen Service zur Präsentation
dieser Event-Daten, sondern überlässt die Zusammenführung mit weiteren
Event-Daten, Filterung und Präsentation den Anbietern, die auf diese
Infrastruktur zugreifen möchten. Die prinzipiellen Möglichkeiten eines solchen
Events Frameworks wurden in mehreren Beispielprojekten demonstriert. 

Der primäre Zugriff erfolgt über Sparql-Anfragen auf einen Sparql-Endpunkt, in
dem die Event-Daten mit weiteren Daten über Veranstalter und
Veranstaltungsorte angereichert und zusammengeführt sind.  Das derzeit
vorliegende einheitliche Format (aka Protokoll) als Ergebnis eines längeren
Entwicklungsprozesses ist weiter unten genauer beschrieben. Fragen der
Weiterentwicklung des Protokolls sind zwischen interessierten Partnern noch
genauer abzustimmen.

\textbf{Prädikate:}
\begin{itemize}\itemsep0pt
\item ld:contactPerson foaf:Person – Ansprechpartner für das Event
\item ical:contact Literal – Kontaktinformation als String
\item ical:description Literal – Beschreibung des Events
\item ld:hatVeranstaltungsort ld:Ort – Ort, an dem das Event stattfindet
\item ld:hatTreffpunkt ld:Treffpunkt – Treffpunkt für das Event
\item ld:hasAddressAddendum Literal – genauere Bezeichnung innerhalb von
  ical:location
\item ical:summary Literal – kurze Beschreibung (max. 100 Zeichen) des Events
\item ical:dtstart, ical:dtend Literale (xsd:date oder xsd:datetime)
\item ld:hatVeranstalter org:Organization – Veranstalter des Events
\item ical:sentBy foaf:Agent – Quelle der Eventinformation
\item ld:hasTag ld:Tag – Tags für das Event
\item ld:zurReihe ld:Projekt – Zuordnung zu einer Veranstaltungsreihe
\item Weitere Orga-Literale wie ld:Kosten
\end{itemize}
\newpage

\section{Leipziger Ecken}
\subsection{Hintergrund}

\emph{Leipziger Ecken} ist eine Stadtteilinitiative, welche die Plattform
\url{https://leipziger-ecken.de} betreibt. Die Plattform basiert auf Drupal
und wird im Wesentlichen von Felix Albroscheit gewartet und weiterentwickelt,
mit dem zunächst eine RDF-Schnittstelle auf einem Dump der Datenbank
entwickelt wurde, die aktuell unter \url{https://leipziger-ecken.de/Data/} auf
der Produktivplattform ausgerollt ist. 

\subsection{Datenmodell}

Im Modell werden die vier Klassen \emph{Akteure}, \emph{Adressen},
\emph{Events} und \emph{Sparten} unterschieden.

\paragraph{Adressen.}
Instanzen dieser Klasse enthalten die Textprädikate plz (Stadt ist immer
Leipzig), strasse, nr, adresszusatz, gps (Geokoordinaten) sowie einen Verweis
auf den Stadtbezirk.  Mit dem Adresszusatz sind die Instanzen also nicht ganz
identisch mit ld:LeipzigerAdresse. Im Transformationsskript werden daraus
(syntaktisch korrekte Vorschläge für) Instanzen von ld:LeipzigerAdresse
generiert.

\paragraph{Akteure.}
In dieser Klasse sind Informationen zur juristischen Person (ld:Akteur), zum
betriebenen Ort (ld:Ort) sowie zu einigen beteiligten Personen (foaf:Person)
zusammengefasst.  Im Transformationsskript wird versucht, diese drei
Bestandteile voneinander zu trennen, wobei die Personen-Referenzen nur bis zu
org:Membership aufgelöst werden können, da Personen über einen Fremdschlüssel
in eine Tabelle referenziert werden, die aus naheliegenden Gründen nicht mit
exportiert wird\footnote{Im Gegensatz zu den anderen Tabellen, die eine
  Erweiterung des Drupal-Datenmodells darstellen, verweisen diese Referenzen
  auf die Drupal-Usertabelle.}. Hier wäre es sinnvoll, diese Angaben
zusätzlich als foaf:Person zu extrahieren und damit die exportierten RDF-Daten
zu ergänzen.

Instanzen dieser Klasse enthalten die Textprädikate name, email, telefon, url,
ansprechpartner,funktion, bild, beschreibung, oeffnungszeiten sowie die
Verweise auf adresse und ersteller. 

\paragraph{Event.}
Die Modellierung folgt der von ld:Event. In der neuen LE-Version sind für
Events nur noch Start- und Endzeit gegeben, die komplexeren Möglichkeiten von
regelmäßig stattfindenden Events wird aktuell -- wie in ld:Event -- nicht
unterstützt.  

Instanzen dieser Klasse enthalten die Textprädikate name, kurzbeschreibung,
bild, url, die Verweise auf ort und ersteller sowie die datetime-Prädikate
\texttt{start\_ts} und \texttt{ende\_ts}

\paragraph{Sparten.}
Instanzen dieser Klasse spannen eine nicht konsolidierte Tagwolke auf, mit
denen die Events kategorisiert werden können. 

Unterschiede zu ld:Event:
\begin{itemize}
\item ical:location verweist nicht auf einen ld:Ort, sondern auf eine
  le:Adresse.
\item ical:creator verweist wieder auf eine Person in der nicht zugänglichen
  Personentabelle.
\item Über le:hatAkteur ist einem Event teilweise ein Akteur zugeordnet.
\item Über le:zurSparte sind einem Event Schlagworte zugeordnet.
\end{itemize}


\section{Nachhaltiges Leipzig}
\subsection{Hintergrund}
\subsection{Datenmodell}

\end{document}
